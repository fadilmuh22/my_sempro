% ----------------------------------------
% Halaman hak cipta
% ----------------------------------------
% \copyright

% ----------------------------------------
% Halaman pengesahan
% ----------------------------------------
% \approval

% ----------------------------------------
% Halaman pernyataan keaslian dan bebas
% plagiarisme
% ----------------------------------------
% \statementcontent{Saya menyatakan bahwa skripsi berjudul ``Pengenalan Emosi Manusia Menggunakan \textit{Log-Gabor Convolutional Networks} Melalui Pendekatan \textit{Facial Region Segmentation}'' ini adalah benar karya saya sendiri. Dan saya tidak melakukan tindakan plagiat yang menyalahi etika dalam karya tulis ilmiah. Apabila saya terbukti bersalah, maka saya bersedia untuk memperbaiki diri, meminta maaf kepada pihak yang bersangkutan dan menanggung setiap sanksi yang berlaku.}
% \statement

% ----------------------------------------
% Halaman ucapan terima kasih/persembahan
% ----------------------------------------
% \acknowledgmentcontent{Untuk ibu, bapak\\dan adik-adikku tercinta.}
% \acknowledgment

% ----------------------------------------
% Kata pengantar
% ----------------------------------------
% \prefacecontent{Segala puji bagi Allah \textit{Subh\=anahu wa Ta’\=ala}, yang dengan nikmat-Nya maka sempurnalah segala kebaikan. Tiada daya dan upaya kecuali hanya dari-Nya. Hanya dengan memohon pertolongan-Nya, penulis dapat menyelesaikan skripsi berjudul ``Pengenalan Emosi Manusia dengan \textit{Log-Gabor Convolutional Networks} Melalui Pendekatan \textit{Facial Region Segmentation}'' ini tepat waktu. Skripsi ini disusun dalam rangka memenuhi sebagian syarat memperoleh gelar Sarjana S-1 Jurusan Ilmu Komputer di Universitas Pendidikan Indonesia.

% Pada kenyataannya, skripsi ini bukan merupakan kredit tersendiri bagi penulis. Melainkan merupakan upaya murni kolaboratif dengan berbagai pihak selama penulis belajar di bangku perkuliahan. Setelah memulai dengan mengucapkan syukur kepada Allah \textit{Subh\=anahu wa Ta’\=ala} di atas segalanya, penulis ingin mengucapkan terima kasih yang tulus kepada kedua pembimbing skripsi ini, bapak Yaya Wihardi dan bapak Wawan Setiawan, karena telah bersedia dengan sepenuh hati membimbing penulis dalam menyelesaikan skripsi ini.

% Sejak diberikan kesempatan oleh bapak Yaya untuk bergabung bersama beliau, bapak Wawan, ibu Enjun Junaeti dan keempat anggota lain dalam riset \textit{smart classroom}, penulis merasa lebih beruntung dari kebanyakan teman-teman lain. Selama melakukan riset bersama-sama, penulis telah banyak belajar dari berbagai tahap yang telah dilalui.

% Penulis ingin mengucapkan terima kasih sebanyak-banyaknya kepada bapak Yaya sebagai mentor terbaik. Selama melakukan bimbingan, beliau telah memberikan sebuah advis kepada penulis bahwa skripsi yang bagus adalah yang selesai. Ketika skripsi itu harus tertunda penyelesaiannya karena ingin serba perfek, maka akan banyak peluang yang mungkin terlewatkan. Terus terang, penulis sangat menyukai bagaimana beliau mengumpulkan setiap mahasiswanya di ruangan yang sama untuk melakukan bimbingan pekanan. Dengan begitu, penulis telah mendapatkan berbagai masukan dan pandangan yang berbeda dari beliau sendiri dan teman-teman saat itu. Selama mengikuti kelas, penulis telah banyak belajar dari beliau khususnya mengenai \textit{computer vision} dan \textit{image processing}. Bagi penulis, beliau termasuk salah satu dosen yang paling cakap dalam mengajar. Sebagai salah satu anggota laboratorium Kecerdasan Buatan dan Robotika, beliau telah meneladani penulis untuk memiliki loyalitas dan kedisplinan yang tinggi.

% Juga terima kasih kepada bapak Wawan yang telah mempercayai penulis sebagai salah satu anggota riset \textit{smart classroom} yang beliau ketuai. Tanpa dukungan yang besar dari beliau, penulis tidak akan memiliki kesempatan untuk dapat bermalam di kampus dan menuntaskan proyek akhir robotika.

% Penulis merasa sangat bersyukur telah diberikan kesempatan dan kepercayaan dalam mengajar sebagai asisten dari ibu Rani Megasari di kelas Basis Data, bapak Yudi Wibisono di kelas Sistem Basis Data, bapak Herbert Siregar di kelas Pemrograman Visual dan bapak Eddy Prasetyo Nugroho di kelas Rekayasa Perangkat Lunak. Terima kasih juga kepada teman-teman yang telah menjadi partner mengajar yang kompeten. Dengan mengajar, penulis tidak hanya mendapatkan pengetahuan yang lebih luas dan mendalam mengenai bahan ajar yang disampaikan, namun juga mendapatkan keluasan untuk meningkatkan kemampuan mengajar dan berbicara di depan kelas. Penulis menjadi mengerti bahwa menempatkan diri bukan sebagai pengajar, akan tetapi sebagai partner bagi para studen, penting dilakukan dalam mengajar di kelas. Hubungan emosional yang baik sedikit banyak mempengaruhi motivasi mereka dalam mengikuti kelas.

% Penulis merasa sangat beruntung telah memiliki dosen-dosen yang istimewa. Sangat menyenangkan mendengarkan mereka saling menceritakan kisah inspiratif satu sama lain. Selain yang sudah disebutkan di atas, penulis ingin berterima kasih lebih khusus kepada ibu Rosa Ariani Sukamto yang telah mengajarkan penulis untuk tidak perlu menjadi orang lain, untuk selalu jujur dalam berlaku dan untuk selalu berjuang juga tidak malu dalam belajar. Juga kepada bapak Yudi yang telah mengajarkan penulis untuk tidak takut berbuat kesalahan dalam belajar dan untuk tidak berhenti belajar sebelum mampu menghasilkan buah karya. Juga kepada bapak Herbert Siregar yang telah mengajarkan penulis untuk selalu belajar memahami sesuatu secara mendasar dan untuk selalu memiliki etos juga etika dalam bekerja. Juga kepada bapak Eddy yang telah mengajarkan penulis untuk selalu menilai sesuatu secara adil juga lugas. Juga kepada ibu Enjun yang telah mengajarkan penulis untuk selalu mengutamakan disiplin dalam berlaku dan untuk selalu menghargai usaha orang lain. Juga kepada dosen-dosen lain yang tidak dapat disebutkan satu per satu. Namun, satu hal yang dapat dipastikan bahwa penulis telah belajar banyak sekali makna dari mereka semua. Jika diperbolehkan, penulis ingin selalu dapat duduk setidaknya satu kali lagi di hadapan mereka untuk mendengarkan dan mencatat beberapa pelajaran terakhir.

% Tidak lupa, penulis juga ingin mengucapkan terima kasih kepada teman-teman yang telah memberikan warna dan menyulap setiap kebersamaan kami dalam belajar di kelas menjadi sangat menyenangkan dan tidak akan pernah tergantikan. Yang telah membuat kenangan dalam ruang-ruang kelas tidak akan pernah sama jika tanpa mereka. Secara istimewa, penulis ingin berterima kasih sebanyak-banyaknya kepada Muhammad Faris Muzakki dan Yahya Firdaus yang telah menjadi partner dalam banyak pekerjaan. Juga kepada Ammar Ashshiddiqi, Reyhan Fikri Dzikriansyah, Teguh Arianto,, Adnan Khairi As., Asep Saepul Ahmad, I. G. N. Agung A. A. W., dan Genta Satria A. P. sebagai teman-teman terdekat penulis dalam menjalani kehidupan di kampus. Juga kepada kakak-kakak tingkat yang telah bersedia menjawab dan memandu penulis dengan begitu tulus dan tanpa pamrih dalam belajar.

% Setiap momen yang penulis habiskan bersama teman-teman seperjuangan adalah menyenangkan dan tidak akan tergantikan. Di sisi lain, setiap momen duduk mencatat dan mengacungkan tangan bertanya mengenai setiap pelajaran yang disampaikan oleh dosen-dosen yang berdedikasi juga tidak akan terlupakan. Bersama dengan mereka semua, kami telah saling berbagi banyak pengetahuan dan cerita.

% Tanpa henti-hentinya, penulis juga bersyukur telah diberikan keluarga yang selalu menjadi orang-orang yang paling pertama dan paling setia dalam mendukung setiap keputusan penulis. Mereka adalah abi Ahmad Djunaedi Sastradinata, umi Desy Rosika Natalia, Ahmad Faaiz Al-Auza'i, Salma Kaisan Syauqi dan Kaisa Rifqa Ghassani. Tanpa doa dan dukungan dari mereka semua, penulis tidak akan mampu berdiri dan melangkah di atas kaki sendiri menuju perjalanan yang penuh dengan kebahagiaan.

% Terakhir, penulis ingin berterima kasih kepada bapak Lala Septem Riza selaku Ketua Departemen Pendidikan Ilmu Komputer Universitas Pendidikan Indonesia, ibu Rani selaku Ketua Program Studi Ilmu Komputer Universitas Pendidikan Indonesia serta semua dosen penguji proposal juga laporan akhir skripsi ini.

% Demikian pengantar ini dibuat dengan sungguh-sungguh. Penulis berharap bahwa pekerjaan ini dapat bermanfaat bagi penulis sendiri dan seluruh pembaca budiman. \textit{At last but not least}, penulis menyatakan secara terbuka untuk menerima segala masukan dalam menyempurnakan skripsi ini.}
% \preface

% ----------------------------------------
% Abstrak
% ----------------------------------------
% \abstractcontent{Pengenalan emosi manusia secara otomatis dapat bermanfaat pada sektor-sektor terkait komputasi afektif. Penelitian ini merupakan penelitian pertama yang mengadopsi teknik \textit{facial region segmentation} (FRS) pada arsitektur \textit{Log-Gabor Convolutional Networks} (Log-GCNs) dalam membangun model menggunakan set data gambar wajah nonfrontal, FER-2013. Dengan menggunakan deteksi \textit{facial landmark}, daerah fitur wajah tertentu dapat disegmentasi menjadi dua-tiga bagian. Setiap bagian dapat dilatih baik secara individu maupun bersamaan menggunakan teknik \textit{network ensemble}, di mana sejumlah arsitektur GCN yang identik tergabung di dalamnya. Hasil eksperimen membuktikan bahwa Log-GCN dengan FRS berhasil mengungguli \emph{baseline} dengan augmentasi data melalui peningkatan akurasi sebesar 6,07\%.}
% \abstractkeywords{Rekognisi emosi; rekognisi ekspresi wajah; FER; segmentasi daerah wajah; \textit{deep convolutional neural network}; jaringan ansambel.}

% \abstractencontent{Automatic recognition of human emotions can be useful in sectors related to affective computing. We believe that it is the first study to adopt facial region segmentation (FRS) techniques on the Log-Gabor Convolutional Networks (Log-GCNs) architecture in order to build a model that using the non-frontal face dataset images, FER-2013. By using facial landmarks detection, certain facial feature areas can be segmented into two-three parts. Each region can be trained either individually or together using network ensemble techniques, where a number of identical GCN architectures are combined. The experimental results prove that Log-GCN with FRS successfully outperformed the baseline with data augmentation through an increase in accuracy of 6.07\%.}
% \abstractenkeywords{Emotion recognition; facial expression recognition; FER; facial region segmentation; deep convolutional neural network; network ensemble.}
% \abstract
