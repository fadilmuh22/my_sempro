\chapter{Pendahuluan}
\section{Latar Belakang}
\qquad Kemajuan teknologi informasi saat ini memungkinkan pengembangan berbagai inovasi dalam bidang pendidikan, termasuk implementasi teknologi berbasis AI untuk meningkatkan interaktivitas dan efektivitas pengajaran. Salah satu aspek penting dalam proses pembelajaran adalah kemampuan seorang pendidik untuk memahami keadaan emosional siswa, yang kerap tercermin melalui ekspresi wajah. Kemampuan ini, jika dapat diukur secara otomatis dan real-time, dapat membantu pendidik dalam menyesuaikan metode pengajaran dan menciptakan suasana belajar yang lebih kondusif.

Aplikasi berbasis Face Recognition dan analisis ekspresi wajah menjadi relevan untuk mengatasi tantangan ini. Teknologi ini memungkinkan sistem untuk mengidentifikasi ekspresi wajah siswa, seperti senang, sedih, atau bingung, yanlg kemudian dapat disampaikan kepada pendidik sebagai indikator untuk menilai suasana kelas secara keseluruhan. Penggunaan komunikasi socket dalam sistem ini, yang memungkinkan transfer data secara langsung dan berkesinambungan antara klien dan server, memfasilitasi proses analisis ekspresi wajah secara real-time dan memastikan data tersampaikan dengan cepat dan akurat \parencite{ogundeyiWebSocketRealTime2019}.

Namun, ada beberapa tantangan teknis dalam mengoptimalkan sistem seperti ini, khususnya dalam hal performa sistem dan kecepatan pemrosesan. Penggunaan socket communication harus dioptimalkan agar dapat menangani lalu lintas data yang besar dengan latensi rendah. Di sisi lain, algoritma analisis ekspresi wajah harus dapat berjalan secara efisien tanpa membebani perangkat pengguna \parencite{phankokkruadRealtimeFaceRecognition2016a}. Penelitian ini bertujuan untuk mengoptimalkan sistem analisis ekspresi wajah pada aplikasi pengajaran berbasis Face Recognition dan socket communication, sehingga dapat memberikan hasil analisis secara real-time dan meminimalkan gangguan teknis yang dapat mengurangi efektivitas pengajaran.

\section{Rumusan Masalah}

 {Berdasarkan latar belakang yang telah dijelaskan, rumusan masalah dalam
  penelitian ini adalah sebagai berikut:}

\begin{enumerate}
	% \tightlist
	\item
	      {Bagaimana mengoptimalkan sistem analisis ekspresi wajah secara
	      Real-Time pada aplikasi pengajaran berbasis Face Recognition?}
	\item
	      {Bagaimana pemanfaatan Socket Communication dapat meningkatkan
	      kecepatan dan keakuratan dalam pengiriman data ekspresi wajah secara
	      Real-Time?}
	\item
	      {Bagaimana mengukur efektivitas optimasi yang diterapkan terhadap
	      kinerja aplikasi?}
\end{enumerate}

\section{Tujuan}

 {Penelitian ini bertujuan untuk:}

\begin{enumerate}
	% \tightlist
	\item
	      {Mengoptimalkan sistem analisis ekspresi wajah pada aplikasi pengajaran berbasis real-time menggunakan Face Recognition dan Socket Communication.}
	\item
	      {Mengidentifikasi tantangan dan mengembangkan solusi dalam optimasi sistem real-time untuk pengajaran interaktif.}
	\item
	      {Mengevaluasi efektivitas dari optimasi yang diterapkan terhadap performa dan kualitas aplikasi dalam mendukung proses pengajaran.}
\end{enumerate}

\section{Manfaat}

 {Adapun manfaat yang diharapkan dari penelitian ini adalah sebagai
  berikut:}

\begin{enumerate}
	% \tightlist
	\item
	      {Menyediakan kontribusi keilmuan di bidang pengenalan wajah dan
	      komunikasi data real-time dalam konteks pendidikan, serta sebagai
	      referensi bagi penelitian selanjutnya.}
	\item
	      {Aplikasi hasil penelitian ini diharapkan dapat membantu pendidik
	      dalam memahami ekspresi dan kondisi emosional siswa secara lebih
	      efektif, sehingga dapat meningkatkan interaktivitas dan kualitas
	      pembelajaran.}
	\item
	      {Menjadi referensi bagi pengembangan sistem serupa yang memerlukan
	      optimasi pengiriman data real-time dan analisis ekspresi wajah dala}{m
	      konteks yang lebih luas.}
\end{enumerate}

\section{Batasan}

 {Untuk memastikan fokus penelitian dan hasil yang terarah, penelitian
  ini memiliki beberapa batasan sebagai berikut:}

\begin{enumerate}
	% \tightlist
	\item
	      {Aplikasi ~akan dibatasi pada penggunaan teknologi face recognition.
	      Menggunakan komunikasi data melalui REST API dan Socket Communication
	      pada aplikasi berbasis web.}
	\item
	      {Pengujian dilakukan dalam lingkungan simulasi kelas kecil, dengan
	      jumlah pengguna yang terbatas.}
	\item
	      {Sistem hanya akan diimplementasikan pada platform WEB Desktop dengan
	      kamera, dan tidak diuji pada perangkat mobile.}
\end{enumerate}
