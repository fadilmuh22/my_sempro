\documentclass{kti}

% ----------------------------------------
% Metadata
% ----------------------------------------
\faculty{KAMPUS UPI DI CIBIRU}
\program{Program Studi REKAYASA PERANGKAT LUNAK}
\title{ANALISIS OPTIMASI SISTEM DETEKSI EKSPRESI WAJAH PADA APLIKASI PENGAJARAN REAL-TIME BERBASIS FACE RECOGNITION DAN SOCKET COMMUNICATION}
% \titleen{Human Emotion Recognition Using \textit{Log-Gabor Convolutional Networks} Through \textit{Facial Region Segmentation} Approach}
\author{Muhamad Fadil}
\email{fadilmuh22@student.upi.edu}
\studentid{2109994}
% \monthsubmit{January}
\yearsubmit{2024}
% \firstsupervisorname{Prof. Dr. H. Wawan Setiawan, M.Kom}
% \firstsupervisorid{196601011991031005}
% \secondsupervisorname{Yaya Wihardi, S.Kom., M.Kom.}
% \secondsupervisorid{198903252015041001}
% \departmentheadname{Dr. Lala Septem Riza, M.T.}
% \departmentheadid{197811262008121001}

% TODO: cek kembali secara menyeluruh
\hyphenation{meng-gu-na-kan seg-men-ta-tion me-ngu-kur me-nye-sat-kan me-nge-na-li pe-nge-na-lan pe-ne-ra-pan ek-strak-si di-man-fa-at-kan eks-pre-si re-le-van ha-ra-pan men-cer-min-kan meng-hu-bung-kan key-board di-u-sul-kan di-la-ku-kan di-gu-na-kan me-nge-na-kan di-sip-lin neu-ron pe-la-ti-han di-ru-mus-kan craw-ling en-hance-ment akan ber-da-sar-kan me-nga-dop-si di-sim-pan me-ne-rap-kan me-ning-kat-kan ke-se-lu-ru-han me-lan-jut-kan di-ten-tu-kan di-per-li-hat-kan ke-sa-la-han dis-gust weighted di-per-lu-kan neu-tral men-ja-jar-kan ka-nan net-work en-sem-ble}

\addbibresource{references.bib}
\emergencystretch=1em

\begin{document}

\frontmatter
% ----------------------------------------
% Halaman judul
% ----------------------------------------
\cover

% ----------------------------------------
% Daftar isi
% ----------------------------------------
\addcontentsline{toc}{chapter}{DAFTAR ISI}
\tableofcontents

% ----------------------------------------
% Daftar tabel
% ----------------------------------------
\listoftables
\addcontentsline{toc}{chapter}{DAFTAR TABEL}

% ----------------------------------------
% Daftar gambar
% ----------------------------------------
\listoffigures
\addcontentsline{toc}{chapter}{DAFTAR GAMBAR}

% ----------------------------------------
% Daftar lampiran
% ----------------------------------------
% [TODO]

\newpage
\mainmatter
% ----------------------------------------
% Isi
% ----------------------------------------
\chapter{Pendahuluan}
\section{Latar Belakang}
\qquad Kemajuan teknologi informasi saat ini memungkinkan pengembangan berbagai inovasi dalam bidang pendidikan, termasuk implementasi teknologi berbasis AI untuk meningkatkan interaktivitas dan efektivitas pengajaran.
Salah satu aspek penting dalam proses pembelajaran adalah kemampuan seorang pendidik untuk memahami keadaan emosional siswa, yang kerap tercermin melalui ekspresi wajah.
Kemampuan ini, jika dapat diukur secara otomatis dan real-time, dapat membantu pendidik dalam menyesuaikan metode pengajaran dan menciptakan suasana belajar yang lebih kondusif.

Aplikasi berbasis Face Recognition dan analisis ekspresi wajah menjadi relevan untuk mengatasi tantangan ini.
Teknologi ini memungkinkan sistem untuk mengidentifikasi ekspresi wajah siswa, seperti senang, sedih, atau bingung, yanlg kemudian dapat disampaikan kepada pendidik sebagai indikator untuk menilai suasana kelas secara keseluruhan.
Penggunaan komunikasi socket dalam sistem ini, yang memungkinkan transfer data secara langsung dan berkesinambungan antara klien dan server, memfasilitasi proses analisis ekspresi wajah secara real-time dan memastikan data tersampaikan dengan cepat dan akurat \parencite{ogundeyiWebSocketRealTime2019}.

Aplikasi pengajaran terpadu yang bernama Emodu yang menggunakan teknologi canggih untuk meningkatkan pengalaman belajar mengajar.
Tech stack yang digunakan dalam pengembangan Emodu meliputi backend dengan NestJS yang berfungsi sebagai bagian utama untuk mengolah data, dan CMS panel dengan Next.js yang memungkinkan siswa dan guru untuk melakukan registrasi.
Guru dapat membuat kelas yang nantinya akan diikuti oleh siswa menggunakan kode unik yang tergenerate.
Dalam kelas tersebut, guru dapat melihat data emosi siswa secara real-time.

Penelitian ini berfokus pada perancangan dan implementasi arsitektur infrastruktur yang tangguh untuk aplikasi EMODU di platform Google Cloud Platform (GCP).
Seiring dengan potensi pertumbuhan pengguna, diperlukan sebuah fondasi infrastruktur yang tidak hanya stabil, tetapi juga mampu beradaptasi secara dinamis.
Oleh karena itu, penelitian ini akan merancang sebuah arsitektur yang mengedepankan tiga pilar utama: skalabilitas otomatis (\textit{auto-scalability}) untuk menangani lonjakan beban pengguna tanpa intervensi manual, pemulihan dari kegagalan (\textit{failure recovery}) untuk memastikan layanan tetap berjalan meskipun terjadi masalah pada salah satu komponen, dan penerapan tanpa henti (\textit{zero-downtime deployment}) yang memungkinkan pembaruan sistem dilakukan tanpa mengganggu aktivitas pengguna.
Dengan menerapkan prinsip-prinsip ini, penelitian bertujuan untuk membangun sebuah sistem yang memiliki ketersediaan tinggi (\textit{high availability}) dan dapat diandalkan untuk mendukung proses belajar mengajar secara berkelanjutan.

Inti dari aplikasi Emodu adalah sebuah ekstensi Google Chrome yang memungkinkan siswa atau peserta meeting untuk join class dan melakukan rekognisi wajah menggunakan library face-api.js yang dijalankan di client.
Teknologi face recognition ini memungkinkan sistem untuk mengidentifikasi ekspresi wajah siswa, seperti senang, sedih, atau bingung, yang kemudian dapat disampaikan kepada pendidik sebagai indikator untuk menilai suasana kelas secara keseluruhan.
Penggunaan komunikasi socket dalam sistem ini memungkinkan transfer data secara langsung dan berkesinambungan antara klien dan server, memfasilitasi proses analisis ekspresi wajah secara real-time dan memastikan data tersampaikan dengan cepat dan akurat.

Namun, ada beberapa tantangan teknis dalam mengoptimalkan sistem seperti ini, khususnya dalam hal performa sistem dan kecepatan pemrosesan.
Penggunaan socket communication harus dioptimalkan agar dapat menangani lalu lintas data yang besar dengan latensi rendah.

\section{Rumusan Masalah}
{Berdasarkan latar belakang yang telah dijelaskan, rumusan masalah dalam penelitian ini adalah sebagai berikut:}

\begin{enumerate}
  \item
    {Bagaimana merancang arsitektur sistem pada Google Cloud Platform (GCP) untuk aplikasi EMODU yang mendukung \textit{high availability}?}
  \item
    {Bagaimana strategi implementasi \textit{auto-scalability} agar sistem dapat menangani fluktuasi beban pengguna secara efisien?}
  \item
    {Bagaimana menerapkan proses \textit{zero-downtime deployment} untuk pembaruan aplikasi EMODU tanpa mengganggu layanan?}
\end{enumerate}

\section{Tujuan}
{Penelitian ini bertujuan untuk:}

\begin{enumerate}
  \item
    {Merancang dan membangun arsitektur sistem yang memiliki \textit{high availability} untuk aplikasi EMODU di lingkungan Google Cloud Platform (GCP).}
  \item
    {Mengimplementasikan mekanisme \textit{auto-scalability} untuk memastikan performa sistem tetap optimal saat terjadi perubahan jumlah pengguna.}
  \item
    {Menerapkan \textit{pipeline deployment} yang memungkinkan pembaruan sistem secara \textit{zero-downtime}.}
\end{enumerate}

\section{Manfaat}

{Adapun manfaat yang diharapkan dari penelitian ini adalah sebagai
berikut:}

\begin{enumerate}
  \item
    {Menghasilkan infrastruktur yang andal, skalabel, dan selalu tersedia untuk aplikasi EMODU, sehingga meningkatkan pengalaman pengguna.}
  \item
    {Menyediakan model arsitektur dan implementasi praktis yang dapat menjadi acuan bagi pengembangan aplikasi lain dengan kebutuhan serupa di platform \textit{cloud}.}
  \item
    {Memberikan kontribusi keilmuan di bidang rekayasa infrastruktur dan DevOps mengenai penerapan prinsip \textit{high availability} dan \textit{auto-scalability} pada platform GCP.}
\end{enumerate}

\section{Batasan}

{Untuk memastikan fokus penelitian dan hasil yang terarah, penelitian
ini memiliki beberapa batasan sebagai berikut:}

\begin{enumerate}
  \item
    {Penelitian berfokus pada perancangan dan implementasi infrastruktur di Google Cloud Platform (GCP), dan tidak melakukan perbandingan dengan platform \textit{cloud} lainnya.}
  \item
    {Studi kasus terbatas pada aplikasi EMODU, tanpa melakukan modifikasi atau optimasi pada kode sumber aplikasi.}
  \item
    {Pengujian sistem dilakukan dalam lingkungan simulasi menggunakan \textit{load testing} untuk mengukur skalabilitas dan ketersediaan, bukan pada pengguna aktif secara masif.}
\end{enumerate}

\chapter{Kajian Pustaka}

\section{State of The Art}

\vspace{-1cm}
\begin{longtable}[t]
{@{}
  |>{\raggedright\arraybackslash}p{(\columnwidth - 6\tabcolsep) * \real{.25}}
  |>{\raggedright\arraybackslash}p{(\columnwidth - 6\tabcolsep) * \real{.25}}
  |>{\raggedright\arraybackslash}p{(\columnwidth - 6\tabcolsep) * \real{.25}}
  |>{\raggedright\arraybackslash}p{(\columnwidth - 6\tabcolsep) * \real{.25}}|
@{}}
\endhead
\endlastfoot

\caption{State of The Art} \\

\hline
\textbf{Judul Penelitian} &
\textbf{Metrik} &
\textbf{Metode Penelitian} &
\textbf{Hasil} 

\\ \hline
{A Real-time Face Recognition for Class Participation Enrollment System Over WebRTC} &
{Akurasi deteksi wajah, ~Attendance reliability, performa Real-Time.} &
{Implementasi pengenalan wajah berbasis WebRTC untuk sistem kehadiran kelas, dengan evaluasi akurasi dan kecepatan respons.} &
{Sistem terbukti efektif untuk mencatat kehadiran kehadiran siswa dan dapat memotivasi siswa untuk hadir di kelas.} 

\\ \hline
{Real Time Face Detection and Facial Expression Recognition: Development and Applications to Human Computer Interaction.} &
{Akurasi deteksi wajah, performa Real-Time.} &
{Deteksi wajah dengan cascade fitur dan pengenalan ekspresi menggunakan Gabor-SVM, diuji pada dataset Cohn-Kanade.} &
{Mencapai akurasi 93\% dalam deteksi ekspresi real-time dan telah diimplementasikan di berbagai platform untuk interaksi manusia-mesin.}

\\ \hline
{Real time Face Detection and Optimal Face Mapping for Online Classes} &
{Akurasi deteksi wajah, ~Attendance reliability, Real-Time Performance.} &
{Perbandingan algoritma Local Binary Pattern Histogram (LBPH) dan Convolutional Neural Network (CNN) untuk pengenalan wajah serta Haar Cascade untuk deteksi wajah.} &
{Algoritma CNN memiliki akurasi 95\%, lebih tinggi dibandingkan LBPH (78\%), menunjukkan keunggulan untuk kehadiran real-time dalam kelas online.}

\\ \hline
{Web Application Development for Biometric Identification System Based on Neural Network Face Recognition} &
{Akurasi deteksi wajah, System Response.} &
{Pengembangan aplikasi web berbasis neural network untuk identifikasi wajah dengan arsitektur MVC dan pengujian performa modul.} &
{Kompleks identifikasi biometrik berhasil diterapkan di lingkungan keamanan seperti bandara dengan akurasi tinggi.} 

\\ \hline
{Web Front-End Realtime Face Recognition Based on TFJS} &
{Latency, Kecepatan deteksi wajah.} &
{Implementasi pengenalan wajah di browser menggunakan TensorFlow.js dan pengujian pada berbagai skenario untuk mengurangi beban server.} &
{Sistem pengenalan wajah Real-Time di sisi Client berhasil menurunkan latensi dan beban Server, menunjukkan respons yang dapat diterima pengguna.} 

\\ \hline
{Web Performance Optimization Techniques for Biodiversity Resource Portal} &
{Efficiency Score, Load Lime, Page Rank, End-to-End Performance Indicators.} &
{Analisis sebelum dan pasca-optimasi menggunakan teknik Web Performance Optimization (WPO), dinilai dengan GTmetrix.} &
{Peningkatan signifikan dalam performa portal, dari Grade F (13\%) menjadi Grade B (82\%) setelah optimasi.}

\\ \hline
{WebSocket in real time application} &
{Latency, Efficiency, Data Transmission Reliability } &
{Analisis performa WebSocket dibandingkan dengan HTTP
long polling dan server-sent events untuk aplikasi real-time.} &
{WebSocket memberikan performa yang lebih baik dalam latensi dan efisiensi jaringan, mendukung pertukaran data dua arah yang lebih halus untuk aplikasi real-time.} 

\\ \hline
{PROPOSED} &
{Memory Usage, System Responses, Real-Time Performance.} &
{Uji coba aplikasi dan web server pada aplikasi web virtual meeting untuk menguji performa.} &
{Menganalisis cara meengoptimalkan kecepatan, keakuratan, dan efektivitas sistem analisis ekspresi wajah berbasis face recognition dan socket communication} 

\\ \hline

\end{longtable}

\newpage

\section{Deteksi Ekspresi}
Deteksi ekspresi wajah adalah proses analisis dan interpretasi ekspresi wajah manusia melalui perangkat lunak untuk mengenali emosi, seperti senang, sedih, bingung, dan fokus. Teknologi ini sering didasarkan pada face recognition yang mengidentifikasi fitur-fitur wajah melalui berbagai teknik seperti pengenalan pola, analisis tekstur, dan deep learning \parencite{archanaRealTimeFace2022}. Dalam aplikasi pengajaran berbasis real-time, deteksi ekspresi wajah memungkinkan pengajar untuk memahami respons siswa dengan cepat, mendukung interaksi yang lebih responsif dan efektif dalam pembelajaran \parencite{archanaRealTimeFace2022}.

\section{Socket Communication}
Socket communication adalah metode yang digunakan untuk memungkinkan transfer data secara real-time antara client dan server. Teknologi ini memanfaatkan protokol TCP/IP atau WebSocket untuk mempertahankan koneksi yang persisten, memungkinkan aplikasi untuk mengirim dan menerima data secara cepat dan efisien. Dalam konteks aplikasi pengajaran berbasis real-time, socket communication berperan penting untuk memastikan bahwa data terkait ekspresi wajah siswa dapat dikirimkan secara Real-Time ke server dan disampaikan kepada pengajar yang memungkinkan minimnya Latency dan Network Traffic di sistem \parencite{ogundeyiWebSocketRealTime2019}.

\section{Ekstensi Pramban}
Ekstensi pramban web adalah aplikasi kecil yang dapat ditambahkan ke peramban untuk memberikan fungsi tambahan atau mengubah perilaku halaman web tertentu \parencite{jinImpactExtensionsBrowser2024}. Ekstensi ini memungkinkan fitur khusus seperti integrasi face recognition, yang berjalan di sisi pengguna untuk memproses data visual secara langsung tanpa memerlukan pengiriman seluruh data gambar ke server. Ekstensi pramban untuk pengajaran dapat mengumpulkan data ekspresi wajah siswa secara otomatis dan mengirimkannya ke server melalui socket communication, menjaga interaksi tetap real-time dan membantu pengajar menilai kondisi siswa lebih akurat.

\section{Performance Testing}
Performance testing adalah serangkaian proses pengujian untuk mengukur stabilitas, kecepatan, dan efisiensi aplikasi dalam berbagai kondisi operasional \parencite{jinImpactExtensionsBrowser2024}. Dalam aplikasi real-time yang menggunakan socket communication dan face recognition, performance testing penting untuk memastikan bahwa pengiriman data, pemrosesan gambar, dan penyampaian hasil tetap cepat dan responsif. Pengujian ini juga berguna untuk mengidentifikasi potensi bottleneck dalam aliran data dan meningkatkan pengoptimalan sistem agar mampu menangani beban pemrosesan dalam situasi penggunaan yang intensif.

\chapter{Metode Penelitian}

\section{Desain Penelitian}

\vspace{-.5cm}
\begin{figure}[h!]
  \includegraphics[
    width=14.5cm,
    height=13.5cm,
  ]{images/drm.png}
  \caption{Design Research Methodology (DRM)}
\end{figure}

Penelitian ini menggunakan pendekatan Design Research Methodology (DRM) untuk mengembangkan solusi teknis secara sistematis.
Tujuan utamanya adalah merancang, mengimplementasikan, dan mengevaluasi arsitektur sistem untuk aplikasi EMODU di Google Cloud Platform (GCP).
Melalui DRM, penelitian berfokus pada perancangan arsitektur yang memenuhi tiga pilar utama: ketersediaan tinggi (\textit{high availability}), skalabilitas otomatis (\textit{scalability}), dan penerapan tanpa henti (\textit{zero-downtime deployment}).

\subsection{Klarifikasi Penelitian}
Tahap ini berfokus pada identifikasi masalah dan penentuan ruang lingkup penelitian.
Masalah utama yang diidentifikasi adalah bagaimana membangun infrastruktur yang andal dan efisien untuk aplikasi EMODU, yang berpotensi mengalami pertumbuhan pengguna dan beban kerja yang fluktuatif.
Ruang lingkup penelitian dibatasi pada perancangan dan implementasi arsitektur di Google Cloud Platform (GCP).
Kajian pustaka akan difokuskan pada konsep-konsep kunci seperti arsitektur \textit{high availability}, mekanisme \textit{scalability} pada GCP (misalnya, Managed Instance Groups), strategi \textit{zero-downtime deployment} (seperti Blue-Green atau Rolling Updates), dan praktik terbaik DevOps untuk CI/CD.

\subsection{Studi Deskriptif I}
Pada tahap ini, dilakukan analisis terhadap arsitektur awal atau baseline dari aplikasi EMODU.
Tujuannya adalah untuk memahami komponen-komponen sistem yang ada (backend NestJS, CMS Next.js, komunikasi socket) dan mengidentifikasi potensi kelemahan infrastruktur.
Analisis ini mencakup identifikasi titik tunggal kegagalan (\textit{single points of failure}), potensi masalah skalabilitas, dan proses deployment yang masih manual.
Hasil dari studi ini akan menjadi dasar untuk perancangan arsitektur yang lebih tangguh dan otomatis.

\subsection{Studi Preskriptif}
Pada tahap ini, dirancang sebuah arsitektur infrastruktur solusi di Google Cloud Platform (GCP) untuk mengatasi masalah yang telah diidentifikasi.
Desain arsitektur akan merinci penggunaan layanan-layanan GCP yang spesifik, seperti Google Compute Engine dengan Managed Instance Groups untuk \textit{scaling}, Cloud Load Balancing untuk distribusi lalu lintas dan \textit{high availability}, serta Cloud Build atau GitHub Actions untuk membangun pipeline CI/CD yang mendukung \textit{zero-downtime deployment}.
Arsitektur ini dirancang secara eksplisit untuk memastikan sistem dapat pulih dari kegagalan, beradaptasi dengan beban pengguna, dan dapat diperbarui tanpa mengganggu layanan.

\subsection{Studi Deskriptif II}
Pada tahap ini, arsitektur yang telah dirancang akan diimplementasikan dan dievaluasi kinerjanya.
Aplikasi EMODU akan di-deploy pada infrastruktur GCP yang baru, kemudian serangkaian pengujian akan dilakukan untuk memvalidasi pencapaian tujuan penelitian.
Evaluasi akan mencakup:
\begin{itemize}
    \item \textbf{Pengujian High Availability:} Melakukan simulasi kegagalan (contoh: menghentikan salah satu instance server) dan mengukur waktu pemulihan sistem secara otomatis.
    \item \textbf{Pengujian scalability:} Menggunakan alat \textit{load testing} seperti JMeter untuk memberikan beban lalu lintas yang tinggi dan memverifikasi bahwa sistem secara otomatis menambah atau mengurangi jumlah instance sesuai dengan beban.
    \item \textbf{Pengujian Zero-Downtime Deployment:} Menjalankan pipeline deployment untuk versi baru aplikasi dan memverifikasi bahwa tidak ada gangguan layanan atau kegagalan permintaan selama proses pembaruan.
\end{itemize}
Hasil dari evaluasi ini akan menunjukkan efektivitas dari arsitektur yang diimplementasikan.

\section{Populasi dan Sampel}
Fokus penelitian ini adalah pada sistem teknis, bukan pada pengguna manusia.
Dengan demikian, \textbf{populasi} dalam penelitian ini adalah keseluruhan kemungkinan konfigurasi arsitektur infrastruktur untuk aplikasi berskala seperti EMODU.
\textbf{Sampel} yang diambil adalah arsitektur spesifik yang dirancang dan diimplementasikan pada Google Cloud Platform dalam penelitian ini.
Objek yang diuji bukanlah partisipan manusia, melainkan kinerja dari komponen-komponen infrastruktur itu sendiri ketika diberikan beban kerja yang disimulasikan.

\section{Alat dan Bahan Penelitian}
\subsection{Alat Penelitian}
Spesifikasi komputer yang digunakan untuk pengembangan dan pengelolaan adalah sebagai berikut:
\begin{itemize}
  \item CPU AMD Ryzen 5 PRO 5650U
  \item RAM 14.94 GB
  \item SSD 256 GB
\end{itemize}

\hspace{-32pt} Perangkat lunak yang akan digunakan dalam penelitian ini meliputi:
\begin{itemize}
  \item Fedora Workstation 41
  \item Node.js, NestJS, ReactJS (untuk aplikasi EMODU)
  \item Docker (untuk kontainerisasi aplikasi)
  \item Google Cloud SDK (gcloud CLI)
  \item Postman (untuk pengujian API)
  \item JMeter (untuk \textit{load testing})
  \item Playwright (untuk pengujian \textit{end-to-end})
  \item Grafana \& Google Cloud Monitoring (untuk observabilitas dan visualisasi metrik)
  \item GitHub Actions (untuk pipeline CI/CD)
  \item Zotero, LaTex, Visual Studio Code
\end{itemize}

\hspace{-32pt} Infrastruktur \textit{cloud} yang digunakan dalam penelitian ini adalah \textbf{Google Cloud Platform (GCP)}. Layanan utama yang akan digunakan meliputi:
\begin{itemize}
  \item \textbf{Google Compute Engine (GCE):} Sebagai penyedia mesin virtual untuk menjalankan aplikasi.
  \item \textbf{Managed Instance Groups (MIGs):} Untuk mengelola grup instance dan menerapkan \textit{scaling}.
  \item \textbf{Cloud Load Balancing:} Untuk mendistribusikan lalu lintas dan menyediakan satu titik akses yang \textit{highly available}.
  \item \textbf{Cloud Build:} Sebagai platform CI/CD untuk otomatisasi build, test, dan deploy.
  \item \textbf{Artifact Registry:} Untuk menyimpan dan mengelola artefak build seperti image Docker.
\end{itemize}

\subsection{Bahan Penelitian}
Bahan penelitian yang digunakan meliputi sumber-sumber literatur teknis, yaitu: artikel jurnal ilmiah, dokumentasi resmi dari Google Cloud Platform, buku, dan panduan praktik terbaik (\textit{best practices}) yang berkaitan dengan arsitektur \textit{high availability}, \textit{scalability}, dan \textit{zero-downtime deployment}.
Selain itu, data utama yang menjadi bahan penelitian adalah data kuantitatif berupa metrik kinerja sistem (misalnya, waktu respons, tingkat kesalahan, utilisasi CPU, jumlah instance) yang dikumpulkan dari hasil pengujian beban dan pemantauan sistem.

\section{Instrumen Penelitian}
Instrumen yang digunakan dalam penelitian ini adalah alat untuk membangun, mengukur, dan mengevaluasi arsitektur sistem.
Instrumen utama adalah \textbf{arsitektur infrastruktur itu sendiri} yang diimplementasikan di GCP.
Untuk mengukur kinerja arsitektur tersebut, digunakan instrumen-instrumen berikut:
\begin{itemize}
    \item \textbf{JMeter dan Playwright:} Digunakan sebagai instrumen \textit{load testing} untuk menghasilkan beban pengguna sintetis dan mengukur metrik kinerja seperti \textit{throughput} dan latensi.
    \item \textbf{Google Cloud Monitoring dan Grafana:} Digunakan sebagai instrumen observabilitas untuk mengumpulkan, memantau, dan memvisualisasikan data metrik dari infrastruktur secara \textit{real-time}, seperti utilisasi CPU, jumlah instance, dan status \textit{health check}.
    \item \textbf{Pipeline CI/CD (GitHub Actions/Cloud Build):} Digunakan sebagai instrumen untuk menguji dan memvalidasi proses \textit{zero-downtime deployment}.
    \item \textbf{Skrip Otomatisasi:} Skrip yang dibuat untuk mensimulasikan kegagalan (fault injection), misalnya mematikan instance secara acak, untuk menguji mekanisme \textit{high availability} dan \textit{-failover}.
\end{itemize}
Kombinasi instrumen ini memungkinkan pengumpulan data yang objektif untuk mengevaluasi keberhasilan implementasi arsitektur.

\section{Analisis Data}
Analisis data dalam penelitian ini bersifat kuantitatif dan berfokus pada metrik kinerja infrastruktur yang telah dikumpulkan.
Data akan dianalisis untuk mengevaluasi setiap tujuan utama penelitian:
\begin{itemize}
    \item \textbf{Analisis Scalability:} Menganalisis grafik korelasi antara peningkatan beban (jumlah pengguna virtual) dengan metrik seperti waktu respons dan jumlah instance server yang aktif. Keberhasilan dinilai dari kemampuan sistem menjaga waktu respons tetap stabil dengan menyesuaikan jumlah instance secara otomatis.
    \item \textbf{Analisis High Availability:} Menganalisis log dan data monitoring untuk mengukur waktu yang dibutuhkan sistem untuk pulih setelah simulasi kegagalan. Keberhasilan diukur dari kecepatan pemulihan dan minimnya tingkat kesalahan selama proses \textit{failover}.
    \item \textbf{Analisis Zero-Downtime Deployment:} Menganalisis metrik ketersediaan (availability) dan tingkat kesalahan (error rate) selama proses deployment. Keberhasilan dinilai jika metrik menunjukkan 100\% ketersediaan dan 0\% kesalahan selama pembaruan berlangsung.
\end{itemize}
Hasil analisis ini akan menjadi dasar untuk menarik kesimpulan mengenai efektivitas arsitektur yang diusulkan dalam mencapai \textit{high availability}, \textit{scalability}, dan \textit{zero-downtime deployment}.

\include{contents/chapter4.tex}
\include{contents/chapter5.tex}

% ----------------------------------------
% Daftar pustaka
% ----------------------------------------
\nocite{*}
\printbibliography[heading=bibintoc, title=DAFTAR PUSTAKA]

% ----------------------------------------
% Lampiran
% ----------------------------------------
% https://tex.stackexchange.com/questions/26732/how-to-get-a-list-of-appendices

\end{document}
